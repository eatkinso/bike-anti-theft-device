\documentclass[]{article}
\usepackage[utf8]{inputenc}
\usepackage[T1]{fontenc}
%\usepackage[euler-digits,euler-hat-accent]{eulervm}
%\usepackage{concrete}
\usepackage{amsmath}
\usepackage{graphicx}
\usepackage{amssymb}
\usepackage{float}
\usepackage{tikz}
\usepackage{pgfplots}
\usepackage[letterpaper, margin=1in]{geometry}
%opening
\title{Concealed Bike Anti-Theft Device}
\author{Elizabeth Atkinson (eatkinso)\\ Srinidhi Raman (nidhim2) \\ Alex Wen (acwen2) }

\begin{document}

\maketitle

\section{Introduction}
\subsection{Problem}

\subsection{Solution Overview}

\subsection{Diagrams}

\subsection{High-Level Requirements}

\begin{enumerate}
	\item If a user tries to remove the bike from a stationary location without the electronic key (RFID tag), the alarm will sound. 
	
	\item The device receives GPS data and records its own position over time. The device also performs rudimentary processing to record its distance traveled and speed. 
	
	\item The device transmits its GPS location data and additional data over LoRa to be received by a base station. 
	
\end{enumerate}

\section{Design}
\subsection{Top-Level Block Diagram}

\subsection{Subsystem Overview}
\subsubsection{Control Subsystem}

Subsystem requirements: 

\begin{enumerate}
	\item The microcontroller buffers GPS data and packetizes it to be sent over LoRa. 
	\item The microcontroller sends appropriate control signals to the GPS module, RFID reader, and the alarm subsystem. 
\end{enumerate}

\subsection{Tolerance Analysis}

\section{Ethics and Safety}





\end{document}
